% Options for packages loaded elsewhere
\PassOptionsToPackage{unicode}{hyperref}
\PassOptionsToPackage{hyphens}{url}
%
\documentclass[
]{scrbook}

\usepackage{amsmath,amssymb}
\usepackage{iftex}
\ifPDFTeX
  \usepackage[T1]{fontenc}
  \usepackage[utf8]{inputenc}
  \usepackage{textcomp} % provide euro and other symbols
\else % if luatex or xetex
  \usepackage{unicode-math}
  \defaultfontfeatures{Scale=MatchLowercase}
  \defaultfontfeatures[\rmfamily]{Ligatures=TeX,Scale=1}
\fi
\usepackage{lmodern}
\ifPDFTeX\else  
    % xetex/luatex font selection
    \setmainfont[]{Noto Sans CJK JP}
\fi
% Use upquote if available, for straight quotes in verbatim environments
\IfFileExists{upquote.sty}{\usepackage{upquote}}{}
\IfFileExists{microtype.sty}{% use microtype if available
  \usepackage[]{microtype}
  \UseMicrotypeSet[protrusion]{basicmath} % disable protrusion for tt fonts
}{}
\makeatletter
\@ifundefined{KOMAClassName}{% if non-KOMA class
  \IfFileExists{parskip.sty}{%
    \usepackage{parskip}
  }{% else
    \setlength{\parindent}{0pt}
    \setlength{\parskip}{6pt plus 2pt minus 1pt}}
}{% if KOMA class
  \KOMAoptions{parskip=half}}
\makeatother
\usepackage{xcolor}
\usepackage[margin=2cm]{geometry}
\usepackage{svg}
\setlength{\emergencystretch}{3em} % prevent overfull lines
\setcounter{secnumdepth}{5}
% Make \paragraph and \subparagraph free-standing
\makeatletter
\ifx\paragraph\undefined\else
  \let\oldparagraph\paragraph
  \renewcommand{\paragraph}{
    \@ifstar
      \xxxParagraphStar
      \xxxParagraphNoStar
  }
  \newcommand{\xxxParagraphStar}[1]{\oldparagraph*{#1}\mbox{}}
  \newcommand{\xxxParagraphNoStar}[1]{\oldparagraph{#1}\mbox{}}
\fi
\ifx\subparagraph\undefined\else
  \let\oldsubparagraph\subparagraph
  \renewcommand{\subparagraph}{
    \@ifstar
      \xxxSubParagraphStar
      \xxxSubParagraphNoStar
  }
  \newcommand{\xxxSubParagraphStar}[1]{\oldsubparagraph*{#1}\mbox{}}
  \newcommand{\xxxSubParagraphNoStar}[1]{\oldsubparagraph{#1}\mbox{}}
\fi
\makeatother

\usepackage{color}
\usepackage{fancyvrb}
\newcommand{\VerbBar}{|}
\newcommand{\VERB}{\Verb[commandchars=\\\{\}]}
\DefineVerbatimEnvironment{Highlighting}{Verbatim}{commandchars=\\\{\}}
% Add ',fontsize=\small' for more characters per line
\usepackage{framed}
\definecolor{shadecolor}{RGB}{241,243,245}
\newenvironment{Shaded}{\begin{snugshade}}{\end{snugshade}}
\newcommand{\AlertTok}[1]{\textcolor[rgb]{0.68,0.00,0.00}{#1}}
\newcommand{\AnnotationTok}[1]{\textcolor[rgb]{0.37,0.37,0.37}{#1}}
\newcommand{\AttributeTok}[1]{\textcolor[rgb]{0.40,0.45,0.13}{#1}}
\newcommand{\BaseNTok}[1]{\textcolor[rgb]{0.68,0.00,0.00}{#1}}
\newcommand{\BuiltInTok}[1]{\textcolor[rgb]{0.00,0.23,0.31}{#1}}
\newcommand{\CharTok}[1]{\textcolor[rgb]{0.13,0.47,0.30}{#1}}
\newcommand{\CommentTok}[1]{\textcolor[rgb]{0.37,0.37,0.37}{#1}}
\newcommand{\CommentVarTok}[1]{\textcolor[rgb]{0.37,0.37,0.37}{\textit{#1}}}
\newcommand{\ConstantTok}[1]{\textcolor[rgb]{0.56,0.35,0.01}{#1}}
\newcommand{\ControlFlowTok}[1]{\textcolor[rgb]{0.00,0.23,0.31}{\textbf{#1}}}
\newcommand{\DataTypeTok}[1]{\textcolor[rgb]{0.68,0.00,0.00}{#1}}
\newcommand{\DecValTok}[1]{\textcolor[rgb]{0.68,0.00,0.00}{#1}}
\newcommand{\DocumentationTok}[1]{\textcolor[rgb]{0.37,0.37,0.37}{\textit{#1}}}
\newcommand{\ErrorTok}[1]{\textcolor[rgb]{0.68,0.00,0.00}{#1}}
\newcommand{\ExtensionTok}[1]{\textcolor[rgb]{0.00,0.23,0.31}{#1}}
\newcommand{\FloatTok}[1]{\textcolor[rgb]{0.68,0.00,0.00}{#1}}
\newcommand{\FunctionTok}[1]{\textcolor[rgb]{0.28,0.35,0.67}{#1}}
\newcommand{\ImportTok}[1]{\textcolor[rgb]{0.00,0.46,0.62}{#1}}
\newcommand{\InformationTok}[1]{\textcolor[rgb]{0.37,0.37,0.37}{#1}}
\newcommand{\KeywordTok}[1]{\textcolor[rgb]{0.00,0.23,0.31}{\textbf{#1}}}
\newcommand{\NormalTok}[1]{\textcolor[rgb]{0.00,0.23,0.31}{#1}}
\newcommand{\OperatorTok}[1]{\textcolor[rgb]{0.37,0.37,0.37}{#1}}
\newcommand{\OtherTok}[1]{\textcolor[rgb]{0.00,0.23,0.31}{#1}}
\newcommand{\PreprocessorTok}[1]{\textcolor[rgb]{0.68,0.00,0.00}{#1}}
\newcommand{\RegionMarkerTok}[1]{\textcolor[rgb]{0.00,0.23,0.31}{#1}}
\newcommand{\SpecialCharTok}[1]{\textcolor[rgb]{0.37,0.37,0.37}{#1}}
\newcommand{\SpecialStringTok}[1]{\textcolor[rgb]{0.13,0.47,0.30}{#1}}
\newcommand{\StringTok}[1]{\textcolor[rgb]{0.13,0.47,0.30}{#1}}
\newcommand{\VariableTok}[1]{\textcolor[rgb]{0.07,0.07,0.07}{#1}}
\newcommand{\VerbatimStringTok}[1]{\textcolor[rgb]{0.13,0.47,0.30}{#1}}
\newcommand{\WarningTok}[1]{\textcolor[rgb]{0.37,0.37,0.37}{\textit{#1}}}

\providecommand{\tightlist}{%
  \setlength{\itemsep}{0pt}\setlength{\parskip}{0pt}}\usepackage{longtable,booktabs,array}
\usepackage{calc} % for calculating minipage widths
% Correct order of tables after \paragraph or \subparagraph
\usepackage{etoolbox}
\makeatletter
\patchcmd\longtable{\par}{\if@noskipsec\mbox{}\fi\par}{}{}
\makeatother
% Allow footnotes in longtable head/foot
\IfFileExists{footnotehyper.sty}{\usepackage{footnotehyper}}{\usepackage{footnote}}
\makesavenoteenv{longtable}
\usepackage{graphicx}
\makeatletter
\newsavebox\pandoc@box
\newcommand*\pandocbounded[1]{% scales image to fit in text height/width
  \sbox\pandoc@box{#1}%
  \Gscale@div\@tempa{\textheight}{\dimexpr\ht\pandoc@box+\dp\pandoc@box\relax}%
  \Gscale@div\@tempb{\linewidth}{\wd\pandoc@box}%
  \ifdim\@tempb\p@<\@tempa\p@\let\@tempa\@tempb\fi% select the smaller of both
  \ifdim\@tempa\p@<\p@\scalebox{\@tempa}{\usebox\pandoc@box}%
  \else\usebox{\pandoc@box}%
  \fi%
}
% Set default figure placement to htbp
\def\fps@figure{htbp}
\makeatother
% definitions for citeproc citations
\NewDocumentCommand\citeproctext{}{}
\NewDocumentCommand\citeproc{mm}{%
  \begingroup\def\citeproctext{#2}\cite{#1}\endgroup}
\makeatletter
 % allow citations to break across lines
 \let\@cite@ofmt\@firstofone
 % avoid brackets around text for \cite:
 \def\@biblabel#1{}
 \def\@cite#1#2{{#1\if@tempswa , #2\fi}}
\makeatother
\newlength{\cslhangindent}
\setlength{\cslhangindent}{1.5em}
\newlength{\csllabelwidth}
\setlength{\csllabelwidth}{3em}
\newenvironment{CSLReferences}[2] % #1 hanging-indent, #2 entry-spacing
 {\begin{list}{}{%
  \setlength{\itemindent}{0pt}
  \setlength{\leftmargin}{0pt}
  \setlength{\parsep}{0pt}
  % turn on hanging indent if param 1 is 1
  \ifodd #1
   \setlength{\leftmargin}{\cslhangindent}
   \setlength{\itemindent}{-1\cslhangindent}
  \fi
  % set entry spacing
  \setlength{\itemsep}{#2\baselineskip}}}
 {\end{list}}
\usepackage{calc}
\newcommand{\CSLBlock}[1]{\hfill\break\parbox[t]{\linewidth}{\strut\ignorespaces#1\strut}}
\newcommand{\CSLLeftMargin}[1]{\parbox[t]{\csllabelwidth}{\strut#1\strut}}
\newcommand{\CSLRightInline}[1]{\parbox[t]{\linewidth - \csllabelwidth}{\strut#1\strut}}
\newcommand{\CSLIndent}[1]{\hspace{\cslhangindent}#1}

\usepackage{fontspec}
\setmainfont{Noto Sans CJK JP}
\makeatletter
\@ifpackageloaded{tcolorbox}{}{\usepackage[skins,breakable]{tcolorbox}}
\@ifpackageloaded{fontawesome5}{}{\usepackage{fontawesome5}}
\definecolor{quarto-callout-color}{HTML}{909090}
\definecolor{quarto-callout-note-color}{HTML}{0758E5}
\definecolor{quarto-callout-important-color}{HTML}{CC1914}
\definecolor{quarto-callout-warning-color}{HTML}{EB9113}
\definecolor{quarto-callout-tip-color}{HTML}{00A047}
\definecolor{quarto-callout-caution-color}{HTML}{FC5300}
\definecolor{quarto-callout-color-frame}{HTML}{acacac}
\definecolor{quarto-callout-note-color-frame}{HTML}{4582ec}
\definecolor{quarto-callout-important-color-frame}{HTML}{d9534f}
\definecolor{quarto-callout-warning-color-frame}{HTML}{f0ad4e}
\definecolor{quarto-callout-tip-color-frame}{HTML}{02b875}
\definecolor{quarto-callout-caution-color-frame}{HTML}{fd7e14}
\makeatother
\makeatletter
\@ifpackageloaded{caption}{}{\usepackage{caption}}
\AtBeginDocument{%
\ifdefined\contentsname
  \renewcommand*\contentsname{Table of contents}
\else
  \newcommand\contentsname{Table of contents}
\fi
\ifdefined\listfigurename
  \renewcommand*\listfigurename{List of Figures}
\else
  \newcommand\listfigurename{List of Figures}
\fi
\ifdefined\listtablename
  \renewcommand*\listtablename{List of Tables}
\else
  \newcommand\listtablename{List of Tables}
\fi
\ifdefined\figurename
  \renewcommand*\figurename{図}
\else
  \newcommand\figurename{図}
\fi
\ifdefined\tablename
  \renewcommand*\tablename{表}
\else
  \newcommand\tablename{表}
\fi
}
\@ifpackageloaded{float}{}{\usepackage{float}}
\floatstyle{ruled}
\@ifundefined{c@chapter}{\newfloat{codelisting}{h}{lop}}{\newfloat{codelisting}{h}{lop}[chapter]}
\floatname{codelisting}{Listing}
\newcommand*\listoflistings{\listof{codelisting}{List of Listings}}
\captionsetup{labelsep=colon}
\makeatother
\makeatletter
\makeatother
\makeatletter
\@ifpackageloaded{caption}{}{\usepackage{caption}}
\@ifpackageloaded{subcaption}{}{\usepackage{subcaption}}
\makeatother

\usepackage{bookmark}

\IfFileExists{xurl.sty}{\usepackage{xurl}}{} % add URL line breaks if available
\urlstyle{same} % disable monospaced font for URLs
\hypersetup{
  pdftitle={調査報告書タイトル},
  pdfauthor={主執筆者名},
  hidelinks,
  pdfcreator={LaTeX via pandoc}}


\title{調査報告書タイトル}
\usepackage{etoolbox}
\makeatletter
\providecommand{\subtitle}[1]{% add subtitle to \maketitle
  \apptocmd{\@title}{\par {\large #1 \par}}{}{}
}
\makeatother
\subtitle{サブタイトル(必要に応じて)}
\author{主執筆者名}
\date{2025-11-03}

\begin{document}
\frontmatter
\maketitle

\renewcommand*\contentsname{Table of contents}
{
\setcounter{tocdepth}{2}
\tableofcontents
}

\mainmatter
\chapter*{エグゼクティブサマリ}\label{ux30a8ux30b0ux30bcux30afux30c6ux30a3ux30d6ux30b5ux30deux30ea}
\addcontentsline{toc}{chapter}{エグゼクティブサマリ}

\begin{tcolorbox}[enhanced jigsaw, colback=white, title=\textcolor{quarto-callout-important-color}{\faExclamation}\hspace{0.5em}{重要ポイント}, arc=.35mm, coltitle=black, rightrule=.15mm, colframe=quarto-callout-important-color-frame, opacityback=0, leftrule=.75mm, opacitybacktitle=0.6, colbacktitle=quarto-callout-important-color!10!white, bottomtitle=1mm, bottomrule=.15mm, left=2mm, titlerule=0mm, toptitle=1mm, toprule=.15mm, breakable]

この報告書は{[}調査期間: 2025-01-01 -
2025-03-31{]}における技術調査に関する包括的な調査結果をまとめたものです。

\end{tcolorbox}

\section{調査の背景と目的}\label{ux8abfux67fbux306eux80ccux666fux3068ux76eeux7684}

{[}調査実施の背景と目的を簡潔に記述{]}

\section{主要な発見事項}\label{ux4e3bux8981ux306aux767aux898bux4e8bux9805}

\begin{enumerate}
\def\labelenumi{\arabic{enumi}.}
\tightlist
\item
  \textbf{発見事項1}: {[}重要な発見事項の概要{]}
\item
  \textbf{発見事項2}: {[}重要な発見事項の概要{]}
\item
  \textbf{発見事項3}: {[}重要な発見事項の概要{]}
\end{enumerate}

\section{推奨事項}\label{ux63a8ux5968ux4e8bux9805}

\begin{tcolorbox}[enhanced jigsaw, colback=white, title=\textcolor{quarto-callout-tip-color}{\faLightbulb}\hspace{0.5em}{提言}, arc=.35mm, coltitle=black, rightrule=.15mm, colframe=quarto-callout-tip-color-frame, opacityback=0, leftrule=.75mm, opacitybacktitle=0.6, colbacktitle=quarto-callout-tip-color!10!white, bottomtitle=1mm, bottomrule=.15mm, left=2mm, titlerule=0mm, toptitle=1mm, toprule=.15mm, breakable]

\begin{enumerate}
\def\labelenumi{\arabic{enumi}.}
\tightlist
\item
  \textbf{短期的提言}: {[}即座に実行すべき事項{]}
\item
  \textbf{中長期的提言}: {[}継続的に取り組むべき事項{]}
\end{enumerate}

\end{tcolorbox}

\chapter{はじめに}\label{ux306fux3058ux3081ux306b}

\section{調査の背景}\label{ux8abfux67fbux306eux80ccux666f}

{[}調査の背景となる状況や問題意識を記述{]}

\section{調査の目的}\label{ux8abfux67fbux306eux76eeux7684}

{[}この調査で明らかにしたいことを明確に記述{]}

\section{調査の範囲と制限事項}\label{ux8abfux67fbux306eux7bc4ux56f2ux3068ux5236ux9650ux4e8bux9805}

{[}調査対象の範囲と制限事項を明記{]}

\begin{tcolorbox}[enhanced jigsaw, colback=white, title=\textcolor{quarto-callout-warning-color}{\faExclamationTriangle}\hspace{0.5em}{調査の制約}, arc=.35mm, coltitle=black, rightrule=.15mm, colframe=quarto-callout-warning-color-frame, opacityback=0, leftrule=.75mm, opacitybacktitle=0.6, colbacktitle=quarto-callout-warning-color!10!white, bottomtitle=1mm, bottomrule=.15mm, left=2mm, titlerule=0mm, toptitle=1mm, toprule=.15mm, breakable]

\begin{itemize}
\tightlist
\item
  調査期間: {[}期間の制約{]}
\item
  情報源: {[}利用可能な情報源の制限{]}
\item
  その他: {[}その他の制約事項{]}
\end{itemize}

\end{tcolorbox}

\section{報告書の構成}\label{ux5831ux544aux66f8ux306eux69cbux6210}

本報告書は以下の構成となっています:

\begin{itemize}
\tightlist
\item
  \textbf{第2章}: 調査方法と手法
\item
  \textbf{第3章}: 調査結果の詳細
\item
  \textbf{第4章}: 分析と考察
\item
  \textbf{第5章}: 結論と提言
\end{itemize}

\chapter{調査方法}\label{ux8abfux67fbux65b9ux6cd5}

\section{調査アプローチ}\label{ux8abfux67fbux30a2ux30d7ux30edux30fcux30c1}

{[}採用した調査方法論を説明{]}

\section{情報源の分類}\label{ux60c5ux5831ux6e90ux306eux5206ux985e}

本調査では以下の情報源を活用しました:

\begin{tcolorbox}[enhanced jigsaw, colback=white, title=\textcolor{quarto-callout-note-color}{\faInfo}\hspace{0.5em}{信頼性の高い情報源(優先順位順)}, arc=.35mm, coltitle=black, rightrule=.15mm, colframe=quarto-callout-note-color-frame, opacityback=0, leftrule=.75mm, opacitybacktitle=0.6, colbacktitle=quarto-callout-note-color!10!white, bottomtitle=1mm, bottomrule=.15mm, left=2mm, titlerule=0mm, toptitle=1mm, toprule=.15mm, breakable]

\begin{enumerate}
\def\labelenumi{\arabic{enumi}.}
\tightlist
\item
  \textbf{行政機関の公式発表}: 政府機関、監督官庁の公式文書
\item
  \textbf{学術論文・研究報告}: 査読付きジャーナル、研究機関の報告書
\item
  \textbf{業界標準・技術仕様書}: 国際標準化機構、業界団体の標準文書
\item
  \textbf{企業公式技術文書}: 信頼性の高い企業の公式技術資料
\end{enumerate}

\end{tcolorbox}

\section{データ収集期間}\label{ux30c7ux30fcux30bfux53ceux96c6ux671fux9593}

\textbf{調査実施期間}: 2025-01-01 - 2025-03-31

\chapter{調査結果}\label{sec-results}

\section{テーマ1:
{[}調査テーマ名{]}}\label{ux30c6ux30fcux30de1-ux8abfux67fbux30c6ux30fcux30deux540d}

\subsection{概要}\label{ux6982ux8981}

{[}テーマ1に関する調査結果の概要を記述{]}

\subsection{詳細分析}\label{ux8a73ux7d30ux5206ux6790}

(詳細な調査結果を記述。事実を述べる際は必ず引用元を明記 山田太郎 and
田中花子 2024)

\begin{tcolorbox}[enhanced jigsaw, colback=white, title=\textcolor{quarto-callout-note-color}{\faInfo}\hspace{0.5em}{関連項目}, arc=.35mm, coltitle=black, rightrule=.15mm, colframe=quarto-callout-note-color-frame, opacityback=0, leftrule=.75mm, opacitybacktitle=0.6, colbacktitle=quarto-callout-note-color!10!white, bottomtitle=1mm, bottomrule=.15mm, left=2mm, titlerule=0mm, toptitle=1mm, toprule=.15mm, breakable]

\hyperref[ux8abfux67fbux65b9ux6cd5]{第2章の調査方法}および
表~\ref{tbl-data} で示すデータと合わせて参照してください。

\end{tcolorbox}

\subsection{データと統計}\label{ux30c7ux30fcux30bfux3068ux7d71ux8a08}

以下の表は主要な調査データを示しています:

\begin{longtable}[]{@{}lll@{}}
\caption{主要調査データ}\label{tbl-data}\tabularnewline
\toprule\noalign{}
項目 & 数値 & 出典 \\
\midrule\noalign{}
\endfirsthead
\toprule\noalign{}
項目 & 数値 & 出典 \\
\midrule\noalign{}
\endhead
\bottomrule\noalign{}
\endlastfoot
指標1 & 75\% & 内閣府 (2024) \\
指標2 & 2.3倍 & 佐藤一郎 and 鈴木次郎 (2024) \\
指標3 & 150件 & 日本産業標準調査会 (2024) \\
\end{longtable}

\section{テーマ2:
{[}調査テーマ名{]}}\label{ux30c6ux30fcux30de2-ux8abfux67fbux30c6ux30fcux30deux540d}

\subsection{概要}\label{ux6982ux8981-1}

{[}テーマ2に関する調査結果の概要{]}

\subsection{技術動向分析}\label{ux6280ux8853ux52d5ux5411ux5206ux6790}

\begin{figure}

\centering{

\pandocbounded{\includesvg[keepaspectratio]{../../sources/diagrams/generated/sample-flow.svg}}

}

\caption{\label{fig-tech-flow}技術導入のフローチャート}

\end{figure}%

図~\ref{fig-tech-flow}
に示すように、技術導入には段階的なプロセスが必要です。

\begin{tcolorbox}[enhanced jigsaw, colback=white, title=\textcolor{quarto-callout-note-color}{\faInfo}\hspace{0.5em}{図表作成方法}, arc=.35mm, coltitle=black, rightrule=.15mm, colframe=quarto-callout-note-color-frame, opacityback=0, leftrule=.75mm, opacitybacktitle=0.6, colbacktitle=quarto-callout-note-color!10!white, bottomtitle=1mm, bottomrule=.15mm, left=2mm, titlerule=0mm, toptitle=1mm, toprule=.15mm, breakable]

このフローチャートは以下のコマンドで生成されました:

\begin{Shaded}
\begin{Highlighting}[]
\ExtensionTok{scripts/generate{-}diagrams.sh}\NormalTok{ mermaid sources/diagrams/mermaid/sample{-}flow.mmd}
\end{Highlighting}
\end{Shaded}

生成されたSVGファイルはInkscapeで高品質編集が可能です:

\begin{Shaded}
\begin{Highlighting}[]
\ExtensionTok{scripts/generate{-}diagrams.sh}\NormalTok{ mermaid sample{-}flow.mmd }\AttributeTok{{-}{-}inkscape}
\end{Highlighting}
\end{Shaded}

\end{tcolorbox}

\chapter{分析と考察}\label{ux5206ux6790ux3068ux8003ux5bdf}

\section{主要な発見事項の分析}\label{ux4e3bux8981ux306aux767aux898bux4e8bux9805ux306eux5206ux6790}

\subsection{発見事項1の詳細分析}\label{ux767aux898bux4e8bux98051ux306eux8a73ux7d30ux5206ux6790}

{[}発見事項1についての詳細な分析を記述{]}

\begin{tcolorbox}[enhanced jigsaw, colback=white, title=\textcolor{quarto-callout-important-color}{\faExclamation}\hspace{0.5em}{考察}, arc=.35mm, coltitle=black, rightrule=.15mm, colframe=quarto-callout-important-color-frame, opacityback=0, leftrule=.75mm, opacitybacktitle=0.6, colbacktitle=quarto-callout-important-color!10!white, bottomtitle=1mm, bottomrule=.15mm, left=2mm, titlerule=0mm, toptitle=1mm, toprule=.15mm, breakable]

\textbf{注意}: ここは考察であることを明示。事実と区別して記述します。

{[}分析者の考察や解釈を記述{]}

\end{tcolorbox}

\textbf{関連項目}: Section~\ref{sec-results}
で示したデータとの関連性については、付録も参照してください。

\subsection{複数の視点からの検討}\label{ux8907ux6570ux306eux8996ux70b9ux304bux3089ux306eux691cux8a0e}

\subsubsection{視点1:
技術的観点}\label{ux8996ux70b91-ux6280ux8853ux7684ux89b3ux70b9}

(異なる立場や理論からの見解を記述 高橋三郎 and 伊藤四郎 2024)

\subsubsection{視点2:
政策的観点}\label{ux8996ux70b92-ux653fux7b56ux7684ux89b3ux70b9}

(別の視点からの見解を記述 中村五郎 and 小林六子 2024)

\section{総合的な評価}\label{ux7dcfux5408ux7684ux306aux8a55ux4fa1}

{[}全体を通じての総合的な評価と考察{]}

\begin{tcolorbox}[enhanced jigsaw, colback=white, title=\textcolor{quarto-callout-tip-color}{\faLightbulb}\hspace{0.5em}{重要な洞察}, arc=.35mm, coltitle=black, rightrule=.15mm, colframe=quarto-callout-tip-color-frame, opacityback=0, leftrule=.75mm, opacitybacktitle=0.6, colbacktitle=quarto-callout-tip-color!10!white, bottomtitle=1mm, bottomrule=.15mm, left=2mm, titlerule=0mm, toptitle=1mm, toprule=.15mm, breakable]

本調査で得られた重要な洞察:

\begin{enumerate}
\def\labelenumi{\arabic{enumi}.}
\tightlist
\item
  {[}洞察1{]}
\item
  {[}洞察2{]}
\item
  {[}洞察3{]}
\end{enumerate}

\end{tcolorbox}

\chapter{結論}\label{ux7d50ux8ad6}

\section{調査結果のまとめ}\label{ux8abfux67fbux7d50ux679cux306eux307eux3068ux3081}

主要な調査結果を以下にまとめます:

\begin{itemize}
\tightlist
\item
  \textbf{結果1}: {[}具体的な調査結果{]}
\item
  \textbf{結果2}: {[}具体的な調査結果{]}
\item
  \textbf{結果3}: {[}具体的な調査結果{]}
\end{itemize}

\section{提言}\label{ux63d0ux8a00-1}

\subsection{短期的提言(1年以内)}\label{ux77edux671fux7684ux63d0ux8a001ux5e74ux4ee5ux5185}

\begin{enumerate}
\def\labelenumi{\arabic{enumi}.}
\tightlist
\item
  \textbf{提言1}: {[}具体的な提言内容{]}

  \begin{itemize}
  \tightlist
  \item
    実施主体: {[}担当者・組織{]}
  \item
    期待効果: {[}期待される効果{]}
  \end{itemize}
\item
  \textbf{提言2}: {[}具体的な提言内容{]}

  \begin{itemize}
  \tightlist
  \item
    実施主体: {[}担当者・組織{]}
  \item
    期待効果: {[}期待される効果{]}
  \end{itemize}
\end{enumerate}

\subsection{中長期的提言(1-3年)}\label{ux4e2dux9577ux671fux7684ux63d0ux8a001-3ux5e74}

\begin{enumerate}
\def\labelenumi{\arabic{enumi}.}
\tightlist
\item
  \textbf{提言3}: {[}中長期的な提言内容{]}

  \begin{itemize}
  \tightlist
  \item
    実施スケジュール: {[}タイムライン{]}
  \item
    リソース要件: {[}必要なリソース{]}
  \end{itemize}
\item
  \textbf{提言4}: {[}中長期的な提言内容{]}

  \begin{itemize}
  \tightlist
  \item
    実施スケジュール: {[}タイムライン{]}
  \item
    リソース要件: {[}必要なリソース{]}
  \end{itemize}
\end{enumerate}

\chapter{今後の課題}\label{ux4ecaux5f8cux306eux8ab2ux984c}

\section{追加調査が必要な領域}\label{ux8ffdux52a0ux8abfux67fbux304cux5fc5ux8981ux306aux9818ux57df}

{[}今回の調査で明らかにできなかった点や、さらなる調査が必要な領域{]}

\begin{tcolorbox}[enhanced jigsaw, colback=white, title=\textcolor{quarto-callout-warning-color}{\faExclamationTriangle}\hspace{0.5em}{調査の限界}, arc=.35mm, coltitle=black, rightrule=.15mm, colframe=quarto-callout-warning-color-frame, opacityback=0, leftrule=.75mm, opacitybacktitle=0.6, colbacktitle=quarto-callout-warning-color!10!white, bottomtitle=1mm, bottomrule=.15mm, left=2mm, titlerule=0mm, toptitle=1mm, toprule=.15mm, breakable]

本調査では以下の点について十分な調査ができませんでした:

\begin{itemize}
\tightlist
\item
  {[}制約1{]}
\item
  {[}制約2{]}
\item
  {[}制約3{]}
\end{itemize}

\end{tcolorbox}

\section{継続的なモニタリング項目}\label{ux7d99ux7d9aux7684ux306aux30e2ux30cbux30bfux30eaux30f3ux30b0ux9805ux76ee}

{[}今後も継続的に観察・調査すべき項目{]}

\chapter*{参考文献}\label{ux53c2ux8003ux6587ux732e}
\addcontentsline{toc}{chapter}{参考文献}

\phantomsection\label{refs}
\begin{CSLReferences}{1}{0}
\bibitem[\citeproctext]{ref-policy2024analysis}
中村五郎, and 小林六子. 2024. {``データガバナンス政策の国際比較分析.''}
In \emph{第30回日本行政学会年次大会論文集}, 112--25. 東京: 日本行政学会.
\url{https://jpas.jp/conference/2024/proceedings/policy-analysis}.

\bibitem[\citeproctext]{ref-research2024analysis}
佐藤一郎, and 鈴木次郎. 2024.
{``Ai技術導入による生産性向上に関する分析.''} AIST-2024-001.
産業技術総合研究所.
\url{https://www.aist.go.jp/reports/2024/ai-productivity-analysis}.

\bibitem[\citeproctext]{ref-government2024report}
内閣府. 2024. {``デジタル変革に関する政府統計報告書.''} 政府統計.
内閣府デジタル庁.
\url{https://www.digital.go.jp/reports/2024/digital-transformation}.

\bibitem[\citeproctext]{ref-cite2024example}
山田太郎, and 田中花子. 2024. {``Sample Research Article Title.''}
\emph{Journal of Technology Research} 15 (3): 45--62.
\url{https://doi.org/10.1000/j.example.2024.001}.

\bibitem[\citeproctext]{ref-industry2024standard}
日本産業標準調査会. 2024. \emph{ソフトウェア品質管理標準 JIS x
25010:2024}. 産業標準. 日本産業標準調査会.
\url{https://www.jisc.go.jp/standards/2024/software-quality}.

\bibitem[\citeproctext]{ref-technical2024perspective}
高橋三郎, and 伊藤四郎. 2024.
{``クラウドネイティブアーキテクチャの技術動向.''}
\emph{情報処理学会論文誌} 65 (4): 234--51.
\url{https://doi.org/10.2197/ipsj.2024.technical.001}.

\end{CSLReferences}

\chapter*{付録}\label{ux4ed8ux9332}
\addcontentsline{toc}{chapter}{付録}

\section{付録A: 用語集}\label{ux4ed8ux9332a-ux7528ux8a9eux96c6}

\begin{longtable}[]{@{}lll@{}}
\caption{主要用語一覧}\label{tbl-glossary}\tabularnewline
\toprule\noalign{}
用語 & 説明 & 初出箇所 \\
\midrule\noalign{}
\endfirsthead
\toprule\noalign{}
用語 & 説明 & 初出箇所 \\
\midrule\noalign{}
\endhead
\bottomrule\noalign{}
\endlastfoot
用語1 & {[}用語の説明{]} & Section~\ref{sec-results} \\
用語2 & {[}用語の説明{]} & {[}該当箇所{]} \\
\end{longtable}

\section{付録B:
詳細データ}\label{ux4ed8ux9332b-ux8a73ux7d30ux30c7ux30fcux30bf}

{[}本文に含めるには詳細すぎるデータや図表を掲載{]}

\section{付録C:
調査ツールとチェックリスト}\label{ux4ed8ux9332c-ux8abfux67fbux30c4ux30fcux30ebux3068ux30c1ux30a7ux30c3ux30afux30eaux30b9ux30c8}

{[}使用した調査ツールやチェックリストなど{]}

\begin{center}\rule{0.5\linewidth}{0.5pt}\end{center}

\begin{tcolorbox}[enhanced jigsaw, colback=white, title=\textcolor{quarto-callout-note-color}{\faInfo}\hspace{0.5em}{文書情報}, arc=.35mm, coltitle=black, rightrule=.15mm, colframe=quarto-callout-note-color-frame, opacityback=0, leftrule=.75mm, opacitybacktitle=0.6, colbacktitle=quarto-callout-note-color!10!white, bottomtitle=1mm, bottomrule=.15mm, left=2mm, titlerule=0mm, toptitle=1mm, toprule=.15mm, breakable]

\begin{itemize}
\tightlist
\item
  \textbf{作成日}: 2025-11-03
\item
  \textbf{最終更新日}: 2025-11-03
\item
  \textbf{バージョン}: 1.0
\item
  \textbf{作成者}: {[}作成者名/組織名{]}
\item
  \textbf{レビュー者}: {[}レビュー者名{]}
\item
  \textbf{承認者}: {[}承認者名{]}
\item
  \textbf{ライセンス}: Apache-2.0
\end{itemize}

\end{tcolorbox}

\section{改訂履歴}\label{ux6539ux8a02ux5c65ux6b74}

\begin{longtable}[]{@{}
  >{\raggedright\arraybackslash}p{(\linewidth - 6\tabcolsep) * \real{0.3333}}
  >{\raggedright\arraybackslash}p{(\linewidth - 6\tabcolsep) * \real{0.1667}}
  >{\raggedright\arraybackslash}p{(\linewidth - 6\tabcolsep) * \real{0.2778}}
  >{\raggedright\arraybackslash}p{(\linewidth - 6\tabcolsep) * \real{0.2222}}@{}}
\caption{改訂履歴}\label{tbl-revisions}\tabularnewline
\toprule\noalign{}
\begin{minipage}[b]{\linewidth}\raggedright
バージョン
\end{minipage} & \begin{minipage}[b]{\linewidth}\raggedright
日付
\end{minipage} & \begin{minipage}[b]{\linewidth}\raggedright
変更内容
\end{minipage} & \begin{minipage}[b]{\linewidth}\raggedright
変更者
\end{minipage} \\
\midrule\noalign{}
\endfirsthead
\toprule\noalign{}
\begin{minipage}[b]{\linewidth}\raggedright
バージョン
\end{minipage} & \begin{minipage}[b]{\linewidth}\raggedright
日付
\end{minipage} & \begin{minipage}[b]{\linewidth}\raggedright
変更内容
\end{minipage} & \begin{minipage}[b]{\linewidth}\raggedright
変更者
\end{minipage} \\
\midrule\noalign{}
\endhead
\bottomrule\noalign{}
\endlastfoot
1.0 & 2025-11-03 & 初版作成 & {[}名前{]} \\
& & & \\
\end{longtable}


\backmatter


\end{document}
